\documentclass[]{article}
%Make sure you do the escape shell thing! https://tex.stackexchange.com/questions/99475/how-to-invoke-latex-with-the-shell-escape-flag-in-texstudio-former-texmakerx 
%this is alt shift f1
%opening
\usepackage[top=25mm,bottom=25mm,left=25mm,right=25mm]{geometry}
\usepackage{amsthm}
\usepackage{amsmath}
\usepackage{minted}
\usepackage{amssymb}
\usepackage{amsfonts}
\usepackage{graphicx}
\usepackage{epstopdf}
\usepackage{url}
\usepackage{biblatex}
%\usepackage{setspace} 
%\usepackage{comment}
%\\\usepackage{hyperref}
%\\\usepackage{cleveref}
\usepackage{subcaption}
\usepackage{braket}
\usepackage[dvipsnames]{xcolor}
\usepackage[utf8]{inputenc}
\usepackage[linesnumbered,ruled]{algorithm2e}
\usepackage{tikz,lipsum,lmodern}
\usepackage[most]{tcolorbox}
\addbibresource{bibliography.bib}
\newtheorem{theorem}{Theorem}[section] 
\newtheorem{lemma}[theorem]{Lemma} 
\newtheorem{proposition}{Proposition}[section]
\newtheorem{cor}[theorem] {Corollary} 
\newtheorem{remark}[theorem] {Remark} 
\newtheorem{ass}[theorem] {Assumptions} 
\theoremstyle{definition}
\newtheorem{definition}[theorem] {Definition} 
\newtheorem{conj}[theorem]{Conjecture} 
\newtheorem{step}{STEP}
\newtheorem{observation}[theorem]{Observation}
\newtheorem{example}[theorem] {Example}
\newtheorem{idea}[theorem]{Idea}
\newtheorem{claim}[theorem]{Claim}
\newcommand{\nn}{\in \mathbb{N}}
\newcommand{\R}{\mathbb{R}}
\newcommand{\N}{\mathbb{N}}
\newcommand{\C}{\mathbb{C}}
\newcommand{\bv}{\bibitem}
\DeclareMathOperator{\Cor}{Cor}

\DeclareMathOperator{\rank}{rank}
\DeclareMathOperator{\Cov}{Cov}
\DeclareMathOperator{\Var}{Var}

\newtheorem{lem}{Lemma}
\newtheorem{thm}{Theorem}

\title{}
\author{Sean Thrasher}
\begin{document}

\maketitle

\begin{abstract}

\end{abstract}

We note that doing freezecore twice, we obtain an incorrect value for energy. When you do freezecore once, you get around $ -7.8 $ hartree, whereas doing it twice, you get something closer to $ 0 $. 

But doing freezecore twice is enormously expensive- it involves $ 10 $ qubits! Doing freezecore twice gets it down to eight qubits. The IBM paper manages to get it down to four! How can we do that without sacrificing significantly the accuracy? 

Furthermore, note that in the IBM paper, they do it not with $ 1.57 $ but with $ 2.5 $ Armstrongs, arguing that the VQEs struggle with that. So I've added that as an option. 

I manually included the $ 4 $-qubit Hamiltonian in the project to see what value it gives us as the lowest energy. 

I just did that, and I got $ (-0.05220316421800277+0j) $. Interesting, so it does seem to be quite different to doing freezecore once. Although we should check that's what we get doing it once for $ 2.5 $ armstrongs. 

Next, I guess we should try a better initial Hamiltonian- maybe one whose spectrum is good. What about the Hartree-Fock energy? 
What is the Hartree-Fock energy then? 

Maybe I got the qubit operator wrong? 


We first 

\printbibliography
\end{document}
