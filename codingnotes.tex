%%%%%%%%%%%%%%%%%%%%%%%%%%%%%%%%%%%%%%%%%%%%%%%%%%%%%%%%%%%%%%%
%
% Welcome to Overleaf --- just edit your LaTeX on the left,
% and we'll compile it for you on the right. If you open the
% 'Share' menu, you can invite other users to edit at the same
% time. See www.overleaf.com/learn for more info. Enjoy!
%
%%%%%%%%%%%%%%%%%%%%%%%%%%%%%%%%%%%%%%%%%%%%%%%%%%%%%%%%%%%%%%%
\documentclass{article}
\usepackage{minted}
\usepackage{xcolor} 
\usepackage{tcolorbox}
\definecolor{LightGray}{gray}{0.9}
\begin{document}

What I've done is I've constructed the following dictionary to allow the user to choose the initial and final hamiltonians. 

Initial: A basic operator whose ground state is the HF state. 

There is something that needs fixing.  $|001\rangle$ can be associated with $IIZ$, but $|011 \rangle$ cannot be associated with $IZZ$. You'd need a minus sign. 
This is too crude an initial Hamiltonian, so a slightly wiser choice for the initial Hamiltonian could be the HF energy times this operator we've constructed. 

So, this is wrong, and I'm about to fix it. We'll delve into how we defined the initial operator. 

\begin{minted}
	[
	frame=lines,
	framesep=2mm,
	baselinestretch=1.2,
	fontsize=\tiny,
	linenos
	]
	{python}
	hamiltonian_methods = {
		'initial': {
			'qiskit_hf': {
				'description': 'Use Qiskit Hartree-Fock method to generate Initial Hamiltonian',
				'generate': lambda molecule, taper, freezecore: Moleculeclass(molecule, taper, freezecore).get_hartreefock_in_pauli()
			},
			'qiskit_hf_and_energy': {
				'description': 'Use Qiskit Hartree-Fock method to generate Initial Hamiltonian',
				'generate': lambda molecule, taper, freezecore: Moleculeclass(molecule, taper, freezecore).get_hartreefock_energy()* Moleculeclass(molecule, taper, freezecore).get_hartreefock_in_pauli()
			},
			
			'paper': {
				'description': 'Use Hartree-Fock from a specific paper (custom implementation)',
				'generate': lambda molecule, taper, freezecore:IBM_LiH_initial
			}
		},
		'final': {
			'qiskit': {
				'description': 'Use final Hamiltonian from a paper-specific method',
				'generate': lambda molecule, taper, freezecore: Moleculeclass(molecule, taper,freezecore).get_qubit_operator()
			},
			'paper': {
				'description': 'Use Qiskit for final Hamiltonian (if different from initial)',
				'generate': lambda molecule, taper, freezecore: IBM_LiH
			}
		}
	}
\end{minted}


This is how we converted into Pauli:

\begin{minted}
	[
	frame=lines,
	framesep=2mm,
	baselinestretch=1.2,
	fontsize=\tiny,
	linenos
	]
	{python}
	def get_hartreefock_in_pauli(self):
		# Get the number of spatial orbitals (i.e., the number of qubits in the mapping)
		problem= self.electronic_structure_problem
		# Get the Hartree-Fock state
		hf_state = HartreeFock(problem.num_spatial_orbitals,problem.num_particles,JordanWignerMapper())
		state_vector= Statevector(hf_state)
		binary_string=state_vector.probabilities_dict()
		binary_string= get_string_from_dict(binary_string)
		pauli_op= create_z_operator_from_binary_string(binary_string)
		return pauli_op
\end{minted}

Now, we've defined an operator whose eigenvalue is zero for the HF state, and one otherwise: 

\begin{minted}
	[
	frame=lines,
	framesep=2mm,
	baselinestretch=1.2,
	fontsize=\tiny,
	linenos
	]
	{python}
	def get_hartreefock_in_projector(self):
		# Get the number of spatial orbitals (i.e., the number of qubits in the mapping)
		problem= self.electronic_structure_problem
		# Get the Hartree-Fock state
		hf_state = HartreeFock(problem.num_spatial_orbitals,problem.num_particles,JordanWignerMapper())
		
		state_vector= Statevector(hf_state)
		state_vector.to_operator()
		# Identity operator (same size as the projector, which is typically 2^n for n qubits)
		identity = Operator(np.eye(projector.num_qubits))
		
		# Identity minus projector
		identity_minus_projector = identity - projector
		
		return identity_minus_projector  # This is the operator I - P
\end{minted}

We've modified the dictionary:

\begin{minted}
	[
	frame=lines,
	framesep=2mm,
	baselinestretch=1.2,
	fontsize=\tiny,
	linenos
	]
	{python}
	hamiltonian_methods = {
		'initial': {
			'qiskit_hf': {
				'description': 'This is a simple Hartree-Fock Hamiltonian.',
				'generate': lambda molecule, taper, freezecore: Moleculeclass(molecule, taper, freezecore).get_hartreefock_in_projector()
			},
			'qiskit_hf_and_energy': {
				'description': 'This is a simple Hartree-Fock Hamiltonian, multiplied by the Hartree-fock energy.',
				'generate': lambda molecule, taper, freezecore: Moleculeclass(molecule, taper, freezecore).get_hartreefock_energy()* Moleculeclass(molecule, taper, freezecore).get_hartreefock_in_projector()
			},
			
			'paper': {
				'description': 'Use Hartree-Fock from a specific paper (custom implementation)',
				'generate': lambda molecule, taper, freezecore:IBM_LiH_initial
			}
		},
		'final': {
			'qiskit': {
				'description': 'Use final Hamiltonian from a paper-specific method',
				'generate': lambda molecule, taper, freezecore: Moleculeclass(molecule, taper,freezecore).get_qubit_operator()
			},
			'paper': {
				'description': 'Use Qiskit for final Hamiltonian (if different from initial)',
				'generate': lambda molecule, taper, freezecore: IBM_LiH
			}
		}
	}
\end{minted}

I'm feeling this is a bit dubious- the Operator doesn't seem defined. So, I'm going to test it. The best way to test it is to ask the alternative run method to output the initial hamiltonian. 

\begin{minted}
	[
	frame=lines,
	framesep=2mm,
	baselinestretch=1.2,
	fontsize=\tiny,
	linenos
	]
	{python}
	def get_hartreefock_in_projector(self):
		# Get the number of spatial orbitals (i.e., the number of qubits in the mapping)
		problem= self.electronic_structure_problem
		# Get the Hartree-Fock state
		hf_state = HartreeFock(problem.num_spatial_orbitals,problem.num_particles,JordanWignerMapper())
		state_vector= Statevector(hf_state)
		state_vector.to_operator()
		# Identity operator (same size as the projector, which is typically 2^n for n qubits)
		identity = Operator(np.eye(projector.num_qubits))
		# Identity minus projector
		identity_minus_projector = identity - projector
		return identity_minus_projector  # This is the operator I - P
\end{minted}
Now, I've fixed a few major bugs with the code. I've simplified the projector. I will use sparse pauli to come up with I-projector. 

\begin{minted}
	[
	frame=lines,
	framesep=2mm,
	baselinestretch=1.2,
	fontsize=\tiny,
	linenos
	]
	{python}
	def get_hartreefock_in_projector(self):
	# Get the number of spatial orbitals (i.e., the number of qubits in the mapping)
	problem= self.electronic_structure_problem
	# Get the Hartree-Fock state
	hf_state = HartreeFock(problem.num_spatial_orbitals,problem.num_particles,JordanWignerMapper())
	state_vector= Statevector(hf_state)
	projector=state_vector.to_operator()
	
	return projector
\end{minted}

We start with running the code, and that is the alternative\_run bit. 

\begin{minted}
	[
	frame=lines,
	framesep=2mm,
	baselinestretch=1.2,
	fontsize=\tiny,
	linenos,highlightlines={ 24 }
	]
	{python}
	from aqc_pqc import AQC_PQC
	from hamiltonian import Hamiltonian 
	from quantum_circuit import QCir
	import networkx as nx
	import numpy as np
	from brute_force import Brute_Force
	from qaoa_circuit import QAOA_Circuit
	from aqc_qaoa import AQC_PQC_QAOA
	from Quantum_Chemistry import Moleculeclass,Solvebynumpy
	from aavqe import *
	from qiskit_algorithms.utils import algorithm_globals
	from qiskit_aer import Aer
	backend = Aer.get_backend('statevector_simulator')
	seed = 3
	number_of_qubits = 8
	steps = 10#Choose number of steps to interpolate from initial to final Hamiltonian
	connectivity = 'nearest-neighbors' #This is the connectivity of the non-parameterized gates in the Hardware-Efficient ansatz
	single_qubit_gates = 'ry'
	entanglement_gates = 'cz'
	layers = 1
	entanglement = 'linear'
	hfstate=Moleculeclass(molecule,taper,freezecore).get_hartreefock()
	myaavqe=My_AAVQE(number_of_qubits,steps,layers,single_qubit_gates,entanglement_gates,entanglement,'qiskit_hf','qiskit')
	myaavqe.alternative_run()
\end{minted}

Here we've initialised My AAVQE. 

Line 6 pulls the appropriate operator given the definition. This needs to be checked- that the initial operator is defined correctly. 
It is used later in the method on line 31, and I got an error on line 34. The method on line 31 takes angles and observables as arguments, each of which gets fed through the alternative\_run method on line 39. Specifically, on line 42. 

\begin{minted}
	[
	frame=lines,
	framesep=2mm,
	baselinestretch=1.2,
	fontsize=\tiny,
	linenos,highlightlines={ 6,31,34 }
	]
	{python}
	def __init__(self, number_of_qubits, steps, layers, single_qubit_gates, entanglement_gates, entanglement,initial_hamiltonian,target_hamiltonian,initial_state=None):
	self.number_of_qubits = number_of_qubits 
	self.initial_state=initial_state   
	self.steps = steps
	self.string_initial_hamiltonian=initial_hamiltonian
	self.initial_hamiltonian=hamiltonian_methods['initial'][initial_hamiltonian]['generate'](molecule,taper,freezecore)
	self.string_final_hamiltonian=target_hamiltonian
	self.offset=0
	self.layers = layers
	self.single_qubit_gates = single_qubit_gates
	self.entanglement_gates = entanglement_gates
	self.entanglement = entanglement
	if self.string_initial_hamiltonian == 'transverse':
	X_tuples = []
	for i in range(number_of_qubits):
	X_tuples.append(('X', [i], -1))
	self.initial_hamiltonian = SparsePauliOp.from_sparse_list([*X_tuples], num_qubits = number_of_qubits)
	elif  self.string_initial_hamiltonian == 'paper':
	self.initial_parameters=[0 for x in range(self.number_of_qubits*(self.layers+1))]
	self.initial_parameters[6]=np.pi
	self.initial_parameters[7]=np.pi
	else:
	self.initial_parameters=[0 for x in range(self.number_of_qubits*(self.layers+1))]
	self.initial_parameters[8]=np.pi
	self.initial_parameters[12]=np.pi
	self.target_hamiltonian=hamiltonian_methods['final'][target_hamiltonian]['generate'](molecule, taper, freezecore)
	print(self.initial_hamiltonian)
	self.qcir = TwoLocal(self.number_of_qubits, self.single_qubit_gates, self.entanglement_gates, self.entanglement, self.layers,initial_state= self.initial_state)
	self.number_of_parameters = len(self.initial_parameters)
	
	def get_expectation_value(self, angles, observable):
	estimator = StatevectorEstimator()
	pub = (self.qcir, observable, angles)
	job = estimator.run([pub])
	result = job.result()[0]
	expectation_value = result.data.evs
	return np.real(expectation_value)
	
	def alternative_run(self):
	lambdas = [i for i in np.linspace(0, 1, self.steps+1)][1:]
	optimal_thetas = self.initial_parameters.copy()
	instantaneous_expectation_value=self.get_expectation_value(optimal_thetas,self.initial_hamiltonian)
	initial_ground_state=self.minimum_eigenvalue(self.initial_hamiltonian)
	energies_aavqe = [instantaneous_expectation_value]
	energies_exact = [initial_ground_state]
	print(f'We start with the optimal angles of the initial hamiltonian: {optimal_thetas}')
	for lamda in lambdas:
	print('\n')
	hamiltonian = self.get_instantaneous_hamiltonian(lamda)
	minimization_object = optimize.minimize(self.get_expectation_value, x0=optimal_thetas, args=(hamiltonian), method='SLSQP')
	optimal_thetas = minimization_object.x
	print(f'We are working on {lamda} where the current optimal point is {optimal_thetas}')
	self.offset=0
	inst_exp_value = self.get_expectation_value(optimal_thetas, hamiltonian) - lamda*self.offset
	energies_aavqe.append(inst_exp_value)
	energies_exact.append(self.minimum_eigenvalue(hamiltonian) - lamda*self.offset)
	print(f'and the instantaneous expectation values is {inst_exp_value}') 
	print(f'and the true expectation value is {self.minimum_eigenvalue(hamiltonian) - lamda*self.offset}')
	plt.plot(energies_aavqe,label='aavqe energy')
	plt.plot(energies_exact,label='true energy')
	plt.legend()
	plt.xlabel('time')
	plt.ylabel('energy (Ha)')
	plt.title(f'{self.string_initial_hamiltonian} and {self.string_final_hamiltonian}')
	plt.show()
	return energies_aavqe
\end{minted}


We start with running the code, and that is the alternative\_run bit. 

\begin{minted}
	[
	frame=lines,
	framesep=2mm,
	baselinestretch=1.2,
	fontsize=\tiny,
	linenos,highlightlines={ 24 }
	]
	{python}
	from aqc_pqc import AQC_PQC
	from hamiltonian import Hamiltonian 
	from quantum_circuit import QCir
	import networkx as nx
	import numpy as np
	from brute_force import Brute_Force
	from qaoa_circuit import QAOA_Circuit
	from aqc_qaoa import AQC_PQC_QAOA
	from Quantum_Chemistry import Moleculeclass,Solvebynumpy
	from aavqe import *
	from qiskit_algorithms.utils import algorithm_globals
	from qiskit_aer import Aer
	backend = Aer.get_backend('statevector_simulator')
	seed = 3
	number_of_qubits = 8
	steps = 10#Choose number of steps to interpolate from initial to final Hamiltonian
	connectivity = 'nearest-neighbors' #This is the connectivity of the non-parameterized gates in the Hardware-Efficient ansatz
	single_qubit_gates = 'ry'
	entanglement_gates = 'cz'
	layers = 1
	entanglement = 'linear'
	hfstate=Moleculeclass(molecule,taper,freezecore).get_hartreefock()
	myaavqe=My_AAVQE(number_of_qubits,steps,layers,single_qubit_gates,entanglement_gates,entanglement,'qiskit_hf','qiskit')
	myaavqe.alternative_run()
\end{minted}

Here we've initialised My AAVQE. 

Line 6 pulls the appropriate operator given the definition. This needs to be checked- that the initial operator is defined correctly. 
It is used later in the method on line 31, and I got an error on line 34. The method on line 31 takes angles and observables as arguments, each of which gets fed through the alternative\_run method on line 39. Specifically, on line 42. 

\begin{minted}
	[
	frame=lines,
	framesep=2mm,
	baselinestretch=1.2,
	fontsize=\tiny,
	linenos,highlightlines={ 6,31,34 }
	]
	{python}
	def __init__(self, number_of_qubits, steps, layers, single_qubit_gates, entanglement_gates, entanglement,initial_hamiltonian,target_hamiltonian,initial_state=None):
	self.number_of_qubits = number_of_qubits 
	self.initial_state=initial_state   
	self.steps = steps
	self.string_initial_hamiltonian=initial_hamiltonian
	self.initial_hamiltonian=hamiltonian_methods['initial'][initial_hamiltonian]['generate'](molecule,taper,freezecore)
	self.string_final_hamiltonian=target_hamiltonian
	self.offset=0
	self.layers = layers
	self.single_qubit_gates = single_qubit_gates
	self.entanglement_gates = entanglement_gates
	self.entanglement = entanglement
	if self.string_initial_hamiltonian == 'transverse':
	X_tuples = []
	for i in range(number_of_qubits):
	X_tuples.append(('X', [i], -1))
	self.initial_hamiltonian = SparsePauliOp.from_sparse_list([*X_tuples], num_qubits = number_of_qubits)
	elif  self.string_initial_hamiltonian == 'paper':
	self.initial_parameters=[0 for x in range(self.number_of_qubits*(self.layers+1))]
	self.initial_parameters[6]=np.pi
	self.initial_parameters[7]=np.pi
	else:
	self.initial_parameters=[0 for x in range(self.number_of_qubits*(self.layers+1))]
	self.initial_parameters[8]=np.pi
	self.initial_parameters[12]=np.pi
	self.target_hamiltonian=hamiltonian_methods['final'][target_hamiltonian]['generate'](molecule, taper, freezecore)
	print(self.initial_hamiltonian)
	self.qcir = TwoLocal(self.number_of_qubits, self.single_qubit_gates, self.entanglement_gates, self.entanglement, self.layers,initial_state= self.initial_state)
	self.number_of_parameters = len(self.initial_parameters)
	
	def get_expectation_value(self, angles, observable):
	estimator = StatevectorEstimator()
	pub = (self.qcir, observable, angles)
	job = estimator.run([pub])
	result = job.result()[0]
	expectation_value = result.data.evs
	return np.real(expectation_value)
	
	def alternative_run(self):
	lambdas = [i for i in np.linspace(0, 1, self.steps+1)][1:]
	optimal_thetas = self.initial_parameters.copy()
	instantaneous_expectation_value=self.get_expectation_value(optimal_thetas,self.initial_hamiltonian)
	initial_ground_state=self.minimum_eigenvalue(self.initial_hamiltonian)
	energies_aavqe = [instantaneous_expectation_value]
	energies_exact = [initial_ground_state]
	print(f'We start with the optimal angles of the initial hamiltonian: {optimal_thetas}')
	for lamda in lambdas:
	print('\n')
	hamiltonian = self.get_instantaneous_hamiltonian(lamda)
	minimization_object = optimize.minimize(self.get_expectation_value, x0=optimal_thetas, args=(hamiltonian), method='SLSQP')
	optimal_thetas = minimization_object.x
	print(f'We are working on {lamda} where the current optimal point is {optimal_thetas}')
	self.offset=0
	inst_exp_value = self.get_expectation_value(optimal_thetas, hamiltonian) - lamda*self.offset
	energies_aavqe.append(inst_exp_value)
	energies_exact.append(self.minimum_eigenvalue(hamiltonian) - lamda*self.offset)
	print(f'and the instantaneous expectation values is {inst_exp_value}') 
	print(f'and the true expectation value is {self.minimum_eigenvalue(hamiltonian) - lamda*self.offset}')
	plt.plot(energies_aavqe,label='aavqe energy')
	plt.plot(energies_exact,label='true energy')
	plt.legend()
	plt.xlabel('time')
	plt.ylabel('energy (Ha)')
	plt.title(f'{self.string_initial_hamiltonian} and {self.string_final_hamiltonian}')
	plt.show()
	return energies_aavqe
\end{minted}

Just to check that the projector is working correctly, and so that we are on track to getting the correct initial hamiltonian, 
we've run this code to check which element the Hamiltonian is projecting onto. It turns out the only non-zero element of the matrix
is the $(17,17)$th element, which is what we expect. The Hartree fock state is indeed $|00010001\rangle$.


\begin{minted}
	[
	frame=lines,
	framesep=2mm,
	baselinestretch=1.2,
	fontsize=\tiny,
	linenos,highlightlines={ 6,31,34 }
	]
	{python}
	myaavqe=My_AAVQE(number_of_qubits,steps,layers,single_qubit_gates,entanglement_gates,entanglement,'qiskit_hf','qiskit',hfstate)
	print(myaavqe.initial_hamiltonian)
	H=myaavqe.initial_hamiltonian
	# Assuming `H` is your operator
	matrix = H.to_matrix()
	# Find non-zero elements in the matrix
	non_zero_indices = np.nonzero(matrix)  # Indices of non-zero elements
	non_zero_values = matrix[non_zero_indices]  # Non-zero values
	print(hfstate)
	# Print the non-zero elements and their positions
	print("Non-zero elements:")
	for idx, value in zip(zip(*non_zero_indices), non_zero_values):
	print(f"Index: {idx}, Value: {value}")
	
	# Assuming `H` is your operator and has been converted to a matrix form
	matrix = H.to_matrix()
	
	# Compute eigenvalues and eigenvectors using numpy.linalg.eig
	eigenvalues, eigenvectors = np.linalg.eig(matrix)
	
	# Find the index of the minimum eigenvalue
	min_index = np.argmin(eigenvalues)
	
	# Get the corresponding minimum eigenvector
	min_eigenvector = eigenvectors[:, min_index]
	
	# Normalize the eigenvector (optional but recommended)
	min_eigenvector = min_eigenvector / np.linalg.norm(min_eigenvector)
	
	# Print the minimum eigenvalue and corresponding eigenvector
	print(f"Minimum eigenvalue: {eigenvalues[min_index]}")
	print(f"Minimum eigenvector: {min_eigenvector}")
	#myaavqe.initial_hamiltonian()
	#myaavqe.alternative_run()
	#print(myaavqe.draw_latex())
\end{minted}

Consider the following 


\newpage

\begin{minted}
	[
	frame=lines,
	framesep=2mm,
	baselinestretch=1.2,
	fontsize=\tiny,
	linenos,highlightlines={ 2,13 }
	]
	{python}
def get_hartreefock_in_pauli(self):
	# Get the number of spatial orbitals (i.e., the number of qubits in the mapping)
	problem = self.electronic_structure_problem
	# Get the Hartree-Fock state
	hf_state = HartreeFock(problem.num_spatial_orbitals, problem.num_particles, JordanWignerMapper())
	
	# Create the statevector for the Hartree-Fock state
	state_vector = Statevector(hf_state)
	
	# Get the probabilities dictionary for the state
	binary_string = state_vector.probabilities_dict()
	
	# Initialize an empty list to store the Pauli operator terms
	Z_tuples = []
	
	# Loop over each binary string in the probabilities dictionary
	for bin_str, _ in binary_string.items():
	# Loop over each bit in the binary string
	for i, bit in enumerate(bin_str):
	if bit == '0':
	Z_tuples.append(('Z', [i], -1))  # Append Z for bit 0
	elif bit == '1':
	Z_tuples.append(('Z', [i], 1))  # Append Z for bit 1
	
	# Convert the list of Pauli operators and coefficients into a SparsePauliOp
	hamiltonian = SparsePauliOp.from_sparse_list([*Z_tuples], num_qubits =  len(Z_tuples))
	# print(f'the binary string is {binary_string}')
	return hamiltonian
\end{minted}
\end{document}