\begin{minted}
        [
        frame=lines,
        framesep=2mm,
        baselinestretch=1.2,
        bgcolor=myblue!20,
        fontsize=\tiny,
        linenos,highlightlines={ 2,13 }
        ]
        {python}
def get_hartreefock_in_pauli(self):
    # Get the number of spatial orbitals (i.e., the number of qubits in the mapping)
    problem = self.electronic_structure_problem
    # Get the Hartree-Fock state
    hf_state = HartreeFock(problem.num_spatial_orbitals, problem.num_particles, JordanWignerMapper())
    
    # Create the statevector for the Hartree-Fock state
    state_vector = Statevector(hf_state)
    
    # Get the probabilities dictionary for the state
    binary_string = state_vector.probabilities_dict()
    
    # Initialize an empty list to store the Pauli operator terms
    Z_tuples = []
    
    # Loop over each binary string in the probabilities dictionary
    for bin_str, _ in binary_string.items():
        # Loop over each bit in the binary string
        for i, bit in enumerate(bin_str):
            if bit == '0':
                Z_tuples.append(('Z', [i], -1))  # Append Z for bit 0
            elif bit == '1':
                Z_tuples.append(('Z', [i], 1))  # Append Z for bit 1
    
    # Convert the list of Pauli operators and coefficients into a SparsePauliOp
    hamiltonian = SparsePauliOp.from_sparse_list([*Z_tuples], num_qubits =  len(Z_tuples))
    # print(f'the binary string is {binary_string}')
    return hamiltonian
\end{minted}
